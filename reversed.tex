% !TEX options=--shell-escape
\documentclass[12pt]{article}
\usepackage[T2A]{fontenc}
\usepackage[utf8]{inputenc}
\usepackage[russian]{babel}
\usepackage{amsmath}
\usepackage{amssymb}
\usepackage{minted}
\usemintedstyle{borland}
\setlength{\parskip}{1.5em}
\DeclareRobustCommand{\divby}{%
  \mathrel{\vbox{\baselineskip.65ex\lineskiplimit0pt\hbox{.}\hbox{.}\hbox{.}}}%
}
\usepackage[left=2cm, right=2cm, top=2cm, bottom=2cm, bindingoffset=0cm]{geometry}
\newtheorem{problem}{Задача}
\newcommand\TheSolution{%
  \textbf{Решение.}\\%
}

\begin{document}
    \begin{problem}
        Напишите функцию, находящую натуральные шестизначные числа, которые при перевороте становятся в 4 раза больше себя.
    \end{problem}
    \TheSolution
    Все шестизначные натуральные числа имеют вид $\overline{abcdef}$, где $a, b, c, d, e, f$ — цифры, причем $a \neq 0$.
    Наше число можно также представить в виде такой суммы
    $$
        \overline{abcdef} = 100000a + 10000b + 1000c + 100d + 10e + f
    $$
    это непосредственно вытекает из поразрадного представления натуральных чисел.

    Перевернем исходное число, тогда получится $\overline{fedcba}$.
    Аналогично запишем его в виде известной суммы
    $$
        \overline{fedcba} = 100000f + 10000e + 1000d + 100c + 10b + a
    $$

    По условию задачи $\overline{fedcba} = 4 \cdot \overline{abcdef}$, это можно записать в виде уравнения относительно всех шести цифр $a, b, c, d, e, f$
    \begin{gather}
        4(100000a + 10000b + 1000c + 100d + 10e + f) = 100000f + 10000e + 1000d + 100c + 10b + a \nonumber \\
        \Longleftrightarrow 400000a + 40000b + 4000c + 400d + 40e + 4f = 100000f + 10000e + 1000d + 100c + 10b + a \nonumber \\
        \Longleftrightarrow 399999a + 39990b + 3900c = 600d + 9960e + 99996f \qquad |\div 3 \nonumber \\
        \Longleftrightarrow \underbrace{133333a + 13330b + 1300c}_A = \underbrace{200d + 3320e + 33332f}_B
    \end{gather}
    Заметим, что $B \divby 2$, значит и $A \divby 2$, но $(13330b + 1300c) \divby 2$ в силу четности обоих коэффициентов, значит выполняется условие $133333a \divby 2 \Leftrightarrow a \divby 2$. Таким образом, раз уж $a$ — положительная четная цифра, то $a \in \{2, 4, 6, 8\}$.

    Оценим $B$ сверху. Это число принимает максимальное значение при $d = e = f = 9$, в этом случае оно равно $200\cdot9 + 3320\cdot9 + 33332\cdot9 = 331668$. То есть получается,
    что $B \leqslant 331668$.

    Заметим, что уже при $a \geqslant 4$ верно $A \geqslant 4\cdot133333 = 533332 > 331668 \geqslant B \Rightarrow A > B$,
    то есть равенство $A = B$ не может быть получено при $a \geqslant 4$, а в силу того, что $a \in \{2, 4, 6, 8\}$, эта цифра определяется однозначно
    $$
        \boxed{a = 2}
    $$

    Подставим эту информацию в уравнение (1), получим
    \begin{gather}
        133333\cdot2 + 13330b + 1300c = 200d + 3320e + 33332f \qquad | \div 2 \nonumber \\
        \Longleftrightarrow \underbrace{133333 + 6665b + 650c}_{A'} = \underbrace{100d + 1660e + 16666f}_{B'}
    \end{gather}
    Имеем: $A' \geqslant 133333$. Возьмем в качестве $d, e$ их максимальные значения ($d = e = 9$), чтобы понять, каким минимальным может быть $f$, чтобы $B'$ удовлетворяло оценке на $A'$. Имеем 
    \begin{gather*}
        B' = 100\cdot9 + 1660\cdot9 + 16666f = 15840 + 16666f \geqslant 133333 \\
        \Longleftrightarrow 16666f \geqslant 133333 - 15840 \\
        \Longleftrightarrow f \geqslant \frac{117493}{16666} = 7\frac{831}{16666} \\
        \Longrightarrow f \in \{8, 9\}
    \end{gather*}
    Иными словами, $f$ не может быть меньше 8. Очевидно, что при меньших значениях $d, e$ цифра $f$ должна быть еще больше, а значит приведенная оценка на $f$ справедлива.
    Из этой оценки следует, что соответствующее цифре $f$ слагаемое $16666f \in \{133328, 149994\}$.

    Исследуем $A'$ и $B'$ на предмет делимости на 5. 
    \begin{itemize}
        \item[$A'$:] Слагаемые $6665b$ и $650c$ делятся на 5, значит $A'$ имеет тот же остаток при делении на 5, что и слагаемое 133333.
        Этот остаток равен 3.
        \item[$B'$:] Слагаемые $100d$ и $1660e$ делятся на 5, значит $B'$ имеет тот же остаток при делении на 5, что и слагаемое $16666f$.
        Этот остаток в зависимости от значений $f$ равен 3 для $f = 8$ или 4 для $f = 9$.
    \end{itemize}
    Раз уж имеет место равенство $A' = B'$, то их остатки при делении на 5 должны быть одинаковыми, это означает, что остаток $B'$ обязан быть равным 3, а это может быть только в случае $f = 8$, тем самым мы однозначно определили значение цифры $f$
    $$
        \boxed{f = 8}
    $$

    Подставим эту информацию в уравнение (2), получим
    \begin{gather}
        133333 + 6665b + 650c = 100d + 1660e + 133328 \nonumber \\
        \Longleftrightarrow 5 + 6665b + 650c = 100d + 1660e \qquad | \div 5 \nonumber \\
        \Longleftrightarrow \underbrace{1 + 1333b + 130c}_{A''} = \underbrace{20d + 332e}_{B''}
    \end{gather}
    Заметим, что $B'' \divby 2$, значит и $A'' \divby 2$, при этом $130c$ — четное слагаемое, а $1$ — нечетное слагаемое.
    Значит четность $A''$ будет обеспечена только в том случае, если слагаемое $1333b$ также является нечетным, то есть $b$ — нечетное. Иными словами, $b \in \{1, 3, 5, 7, 9\}$.

    Заметим, что уже при $b \geqslant 3$ имеет место оценка $A'' \geqslant 1 + 1333\cdot3 = 4000$. Но даже при максимальных $d$ и $e$ ($d = e = 9$) $B''$ принимает значение $B''_{\max} = 20\cdot9 + 332\cdot9 = 3168 < 4000 \leqslant A''$. То есть при $b \geqslant 3$ равенство $A'' = B''$ не реализуется, это приводит к тому, что цифра $b$ также определяется однозначно
    $$
        \boxed{b = 1}
    $$

    Подставим эту информацию в уравнение (3), получим
    \begin{gather}
        1 + 1333 + 130c = 20d + 332e \nonumber \\
        \Longleftrightarrow 1334 + 130c = 20d + 332e \qquad | \div 2 \nonumber \\
        \Longleftrightarrow \underbrace{667 + 65c}_{A'''} = \underbrace{10d + 166e}_{B'''}
    \end{gather}
    $B''' \divby 2 \Leftrightarrow A''' \divby 2 \Leftrightarrow (667 + 65c) \divby 2$, значит в силу нечетности 667, $65c$ — нечетное, то есть $c$ — нечетное.
    Иными словами, $c \in \{1, 3, 5, 7, 9\}$. Выразим цифру $e$ из уравнения (4)
    $$
        e = \frac{667 + 65c - 10d}{166}
    $$
    Оценим ее сверху и снизу для разных значений $c$ и попробуем найти значение цифры $d$:
    \begin{itemize}
        \item[$c = 1$:]
            \begin{gather*}
                e = \frac{732 - 10d}{166} = \frac{366 - 5d}{83} \\
                \Longrightarrow 3\frac{72}{83} = \frac{366 - 5\cdot9}{83} \leqslant e \leqslant \frac{366}{83} = 4\frac{34}{83} \\
                \Longrightarrow e = 4 \\
                \Longrightarrow 5d = 366 - 83e = 366 - 83\cdot4 = 34 \Rightarrow \varnothing
            \end{gather*}
        \item[$c = 3$:]
            \begin{gather*}
                e = \frac{862 - 10d}{166} = \frac{431 - 5d}{83} \\
                \Longrightarrow 4\frac{54}{83} = \frac{431 - 5\cdot9}{83} \leqslant e \leqslant \frac{431}{83} = 5\frac{16}{83} \\
                \Longrightarrow e = 5 \\
                \Longrightarrow 5d = 431 - 83e = 431 - 83\cdot5 = 16 \Rightarrow \varnothing
            \end{gather*}
        \item[$c = 5$:]
            \begin{gather*}
                e = \frac{992 - 10d}{166} = \frac{496 - 5d}{83} \\
                \Longrightarrow 5\frac{36}{83} = \frac{496 - 5\cdot9}{83} \leqslant e \leqslant \frac{496}{83} = 5\frac{81}{83} \\
                \Longrightarrow \varnothing
            \end{gather*}
        \item[$c = 7$:]
            \begin{gather*}
                e = \frac{1122 - 10d}{166} = \frac{561 - 5d}{83} \\
                \Longrightarrow 6\frac{18}{83} = \frac{561 - 5\cdot9}{83} \leqslant e \leqslant \frac{561}{83} = 6\frac{63}{83} \\
                \Longrightarrow \varnothing
            \end{gather*}
        \item[$c = 9$:]
            \begin{gather*}
                e = \frac{1252 - 10d}{166} = \frac{626 - 5d}{83} \\
                \Longrightarrow 7 = \frac{626 - 5\cdot9}{83} \leqslant e \leqslant \frac{626}{83} = 7\frac{45}{83} \\
                \Longrightarrow e = 7 \\
                \Longrightarrow 5d = 626 - 83e = 626 - 83\cdot7 = 45 \Leftrightarrow d = 9
            \end{gather*}
    \end{itemize}
    Таким образом, из всех значений цифры $c$ подошло только значение $c = 9$, причем в этом случае однозначно задаются оставшиеся цифры $e = 7$ и $d = 9$
    $$
        \boxed{c = 9,\quad e = 7,\quad d = 9}
    $$

    Подводя итоги, понимаем, что исходя из условий задачи нам удалось получить единственный набор цифр $a, b, c, d, e, f$ исходного числа
    $$
        \begin{cases}
            a = 2, \\
            b = 1, \\
            c = 9, \\
            d = 9, \\
            e = 7, \\
            f = 8  
        \end{cases}
    $$
    Итак, единственным числом, удовлетворяющим условию задачи, является $\overline{abcdef} = 219978$.

    Программка, решающая задачу (язык Си) 

    \begin{minted}{c}
    #include <stdio.h>

    int reversed_fourth(void);

    int main(void)
    {
        printf("Number: %d", reversed_fourth());
        return 0;
    }

    int reversed_fourth(void) { return 219978; }
    \end{minted}

    \textit{Примечание}. внимательный читатель заметил, что отбор цифр из имеющихся сужающих оценок выполнялся по принципу: если из всего множества вариантов остался всего один, то он подходит. Очевидно, что в процессе решения можно было случайно столкнуться с ситуацией, при которой чисел, которые мы искали, вовсе не существует, однако при постановке задачи подобного сорта уже заранее предусматривается, что такие числа есть, а значит такой принцип применять имеет смысл. При надобности оставшийся вариант всегда можно проверить имеющимися сужающими оценками.
\end{document}